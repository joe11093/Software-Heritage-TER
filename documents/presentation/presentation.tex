\documentclass{beamer}
\usepackage[utf8]{inputenc}
\usepackage[T1]{fontenc}
\usepackage[francais]{babel}
\usepackage{amssymb}
\usetheme{Singapore}

\title{Software Heritage}
\subtitle{HMIN201 - M1 TER}
\author{Bachar RIMA \and Joseph SABA \and Tasnim SHAQURA}
\institute[UM]{M1 AIGLE}
\titlegraphic{
  \includegraphics[scale=0.4]{images/umLogo.png}
  \hspace{\fill}
  \includegraphics[scale=0.25]{images/fdsLogo.jpg}
}
\date{3 juin 2019}

\begin{document}

%titre
\begin{frame}
  \titlepage
\end{frame}

%sommaire
\begin{frame}{Sommaire}
  \tableofcontents[hideallsubsections]
\end{frame}

%Section "Introduction"
\section{Introduction}

  %Sous-section "Description de Software Heritage"
  \subsection{Description de Software Heritage}
    \begin{frame}{Description de Software Heritage}{Introduction}
    \end{frame}

  %Sous-section "Contexte du TER"
  \subsection{Contexte du TER}
    \begin{frame}{Contexte du TER}{Introduction}
    \end{frame}

%Section "Problématique"
\section{Problématique}

  %Sous-section "Le diaspora du code source"
  \subsection{Le diaspora du code source}
    \begin{frame}{Le diaspora du code source}{Problématique}
    \end{frame}

  %Sous-section "La fragilité du code source"
  \subsection{La fragilité du code source}
    \begin{frame}{La fragilité du code source}{Problématique}
    \end{frame}

  %Sous-section "Notre contribution"
  \subsection{Notre contribution}
    \begin{frame}{Notre contribution}{Problématique}
    \end{frame}

%Section "Analyse"
\section{Analyse}
  \begin{frame}{Analyse}
  \end{frame}

  %Sous-section "Fonctionnement de Software Heritage"
  \subsection{Fonctionnement de Software Heritage}
    \begin{frame}{Fonctionnement de Software Heritage}{Analyse}
    \end{frame}

    %Sous-sous-section "Modèle des données"
    \subsubsection{Modèle des données}
      \begin{frame}{Modèle des données}{Fonctionnement de Software Heritage}
      \end{frame}

    %Sous-sous-section "Architecture et flot de données"
    \subsubsection{Architecture et flot de données}
      \begin{frame}{Architecture et flot de données}{Fonctionnement de Software Heritage}
      \end{frame}

    %Sous-sous-section "L'archive de Software Heritage"
    \subsubsection{L'archive de Software Heritage}
      \begin{frame}{L'archive de Software Heritage}{Fonctionnement de Software Heritage}
      \end{frame}
  %Sous-section "Méthodologie"
  \subsection{Méthodologie}
    \begin{frame}{Méthodologie}{Analyse}
    \end{frame}

%Section "Conception"
\section{Conception}
  \begin{frame}{Conception}
  \end{frame}

%Section "Implémentation"
\section{Implémentation}
  \begin{frame}{Implémentation}
  \end{frame}

%Section "Résultats"
\section{Résultats}
  \begin{frame}{Résultats}
  \end{frame}

%Section "Conclusion"
\section{Conclusion}
  \begin{frame}{Conclusion}
  \end{frame}

  %Sous-section "Difficultés rencontées"
  \subsection{Difficultés rencontées}
    \begin{frame}{Difficultés rencontées}{Conclusion}
    \end{frame}

  %Sous-section "Perspectives"
  \subsection{Perspectives}
    \begin{frame}{Perspectives}{Conclusion}
    \end{frame}

\end{document}
