\documentclass[12pt,a4paper]{report}

%----------------------------------------------------------------------------------------
%   PACKAGES
%----------------------------------------------------------------------------------------
\usepackage[francais]{babel} %French language package
\usepackage[utf8]{inputenc} %UTF8
\usepackage[T1]{fontenc} %For the acute french accents
\usepackage[pdftex]{graphicx} %To add figures in the document
\usepackage{hyperref} % To make hyperlinks in the document
\usepackage{amsthm} % To add mathematical symbols

\begin{document}
%----------------------------------------------------------------------------------------
%   TITLE PAGE
%----------------------------------------------------------------------------------------
\begin{titlepage}
\newcommand{\HRule}{\rule{\linewidth}{0.5mm}} % Defines a new command for the horizontal lines
\center

%----------------------------------------------------------------------------------------
%   LOGOS SECTION
%----------------------------------------------------------------------------------------
\includegraphics[scale=0.5]{images/umLogo.png} % Université de Montpellier Logo
\hspace{\fill}
\includegraphics[scale=0.25]{images/fdsLogo.jpg} % Faculté de Sciences Logo

%----------------------------------------------------------------------------------------
%   HEADING SECTIONS
%----------------------------------------------------------------------------------------
\textsc{\LARGE M1 Informatique AIGLE}\\[1cm]
\textsc{\Large \textbf{HMIN201}}\\[0.25cm]
\textsc{\large M1 TER}\\[0.5cm]

%----------------------------------------------------------------------------------------
%   TITLE SECTION
%----------------------------------------------------------------------------------------
\HRule \\[0.4cm]
{ \huge \bfseries TER Software Heritage}\\[0.4cm]
{ \Large \bfseries Rapport Final}\\[0.4cm]
\HRule \\[0.5cm]

%----------------------------------------------------------------------------------------
%   AUTHORS AND SUPERVISORS SECTION
%----------------------------------------------------------------------------------------
{ \huge \bfseries Groupe \textsc{Bajonim}}\\[0.4cm]
\begin{minipage}{0.4\textwidth}
\centering \small
\textbf{Bachar \textsc{Rima}}, \\ \href{mailto:bachar.rima@etu.umontpellier.fr}{bachar.rima@etu.umontpellier.fr}\\ % Student
\textbf{Joseph \textsc{Saba}}, \\ \href{mailto:joseph.saba@etu.umontpellier.fr}{joseph.saba@etu.umontpellier.fr}\\ % Student
\textbf{Tasnim \textsc{Shaqura}}, \\ \href{mailto:tasnim.shaqura@etu.umontpellier.fr}{tasnim.shaqura@etu.umontpellier.fr}\\ % Student
\end{minipage} \\[0.8cm]

\begin{minipage}[b]{0.4\textwidth}
\begin{flushleft} \large
\emph{Encadrant:} \\
Jessie \textsc{Carbonnel} % Academic Supervisor
\end{flushleft}
\end{minipage}
~
\begin{minipage}[b]{0.4\textwidth}
\begin{flushright} \large
\emph{Responsable de l'UE:} \\
Mattieu \textsc{Lafourcade} % UE Supervisor
\end{flushright}
\end{minipage}\\[1.5cm]

%----------------------------------------------------------------------------------------
%   DATE SECTION
%----------------------------------------------------------------------------------------
{\large 27 mai 2019}\\[1cm]
\hspace{\fill}
\vfill % Fill the rest of the page with whitespace
\end{titlepage}

%----------------------------------------------------------------------------------------
%   INTRODUCTION
%----------------------------------------------------------------------------------------
\tableofcontents
\chapter{Introduction}
	\section{Description de Software Heritage}
	
	\section{Contexte du TER}
	\section{Plan du rapport}

\chapter{Problématique}
\section{La diaspora du code source}
\section{La fragilité du code source}
\section{Software Heritage en tant que solution}
	Current status et roadmap de SWH
\section{Notre contribution}

\chapter{Analyse}
\section{Fonctionnement de Software Heritage}
	\subsection{Modèle des données}
		\subsubsection{Plateformes d'hébergement}
		\subsubsection{Artéfacts logiciels}
		\subsubsection{Informations sur la provenance des données}
		\subsubsection{Structure des données}
	\subsection{Architecture et flot des données}
		\subsubsection{Flot d'ingestion des données}
		\subsubsection{Listing}
		\subsubsection{Loading}
		\subsubsection{Scheduling}
	\subsection{L'archive}
		\subsubsection{Stockage des noeuds BLOB de l'archive}
				les IDs, les fichiers
		\subsubsection{Stockage des autres noeuds des archives}
				chemins, repertoires, snapshot, revisions, releases\\				
				postgres, etc
		\subsubsection{Stockage haché des objets}
		\subsubsection{Mise en mirroir des noeuds}
		\subsubsection{Politique de rétention}
		\subsubsection{Récupération automatique des objects corrompus}
\section{Méthodologie}		
	Sourceforge sitemap, api\\
	Launchpad api, client\\
	Analyzing the listers (bitbucket, gitlab, github, eclipse, LIRMM, OpenHub, 			Assembla, GNU savannah\\
	heritage, injection de dependances\\
	conclusion: on adoptera une strategie pour definir un lister, loader ou autre\\
\section{Planning Prévisionnel}
\chapter{Conception}
	design de la solution proposée (diagrammes + explications)
\chapter{Implémentation}
	les technos qu'on a utilisé\\
	bibliotheques\\
	Outils (e.g. XML parsers)\\
	Launchpad client
	
\chapter{Résultats}
	pull request?
\chapter{Conclusion}
	\section{Planning final}
	\section{Difficultés rencontrées}
	\section{Perspectives}
	\section{Bilan et apports du TER}	
annexes\n
\t resumés \t code	
\chapter*{Bibliographie}
\end{document}
